\section{Đánh Giá Hiệu Suất}\label{sec2b}

Phần này trình bày đánh giá toàn diện về hệ thống Học Tăng Cường Sâu (DRL) phân cấp đề xuất 
cho điều khiển đèn giao thông thông minh. Phương pháp đánh giá bao gồm các thí nghiệm dựa trên 
mô phỏng, phân tích so sánh với các phương pháp cơ sở và phân tích tác động của độ phức tạp giao thông 
đến hiệu suất hệ thống để xác thực hiệu quả của cách tiếp cận đa nút giao thông được phối hợp.

\subsection{Thiết Lập Thí Nghiệm}\label{subsec2b-1}

\subsubsection{Môi Trường Mô Phỏng}

Đánh giá hiệu suất được thực hiện sử dụng SUMO (Simulation of Urban Mobility) phiên bản 1.22.0, 
một nền tảng mô phỏng giao thông vi mô cung cấp động lực xe thực tế và mô hình hóa luồng giao thông. 
SUMO cho phép kiểm soát chính xác các tham số giao thông và hỗ trợ mạng lưới đường đô thị phức tạp 
cần thiết cho đánh giá toàn diện hệ thống điều khiển giao thông.

Môi trường mô phỏng tích hợp các cấu hình chính sau:

\textbf{Topology Mạng:} Một mạng giao thông được phối hợp bao gồm 4 nút giao thông, mỗi nút có bốn 
hướng tiếp cận với cấu hình làn đường tiêu chuẩn. Thiết lập này đại diện cho một cụm nút giao thông 
đô thị điển hình phù hợp cho phân tích phối hợp đa nút giao thông và cung cấp độ phức tạp đủ để đánh giá 
lợi ích đồng bộ hóa.

\textbf{Nhu Cầu Giao Thông:} Bốn kịch bản độ phức tạp giao thông riêng biệt được triển khai để đánh giá 
hiệu suất hệ thống trên các điều kiện khác nhau. Các kịch bản dao động từ giao thông thấp ở 300 xe/giờ, 
phục vụ như kịch bản học tập cơ sở, đến giao thông trung bình ở 600 xe/giờ đại diện cho độ phức tạp vừa phải. 
Điều kiện giao thông cao được mô phỏng ở 900 xe/giờ để kiểm tra các kịch bản thách thức, trong khi điều kiện 
giờ cao điểm ở 1.200 xe/giờ đại diện cho các kịch bản độ phức tạp tối đa cho đánh giá hệ thống toàn diện.

\textbf{Động Lực Xe:} Mô hình theo xe Krauss được sử dụng với các tham số tăng tốc/giảm tốc được hiệu chỉnh 
cho điều kiện lái xe đô thị. Tốc độ tối đa được thiết lập ở 50 km/h cho đường chính với các hồ sơ tăng tốc 
thực tế và khoảng cách theo an toàn.

\subsubsection{Cấu Hình Huấn Luyện}

Hệ thống DRL sử dụng kiến trúc Deep Q-Network (DQN) với các đặc tả sau:

\textbf{Kiến Trúc Mô Hình:} Hệ thống sử dụng thuật toán Deep Q-Network (DQN) với trải nghiệm replay 
để học tập mạnh mẽ. Không gian trạng thái bao gồm 80 chiều đại diện cho 8 nhóm làn đường với 10 ô 
không gian mỗi nhóm, nắm bắt chi tiết sự chiếm dụng nút giao thông. Không gian hành động bao gồm 4 hành động: 
xanh Bắc-Nam, xanh Đông-Tây, rẽ trái Bắc-Nam và rẽ trái Đông-Tây. Kiến trúc mạng tuân theo cấu hình lớp 
kết nối đầy đủ 80 → 400 → 400 → 400 → 400 → 4 với các hàm kích hoạt ReLU.

\textbf{Tham Số Huấn Luyện:} Giao thức huấn luyện liên quan đến 150 tập mỗi tác nhân với tốc độ học 0.001 
sử dụng bộ tối ưu hóa Adam. Hệ số chiết khấu được thiết lập ở γ = 0.95 để cân bằng phần thưởng ngay lập tức 
và tương lai. Mỗi cập nhật xử lý các batch 32 chuyển đổi, trong khi khám phá $\epsilon$-tham lam suy giảm 
từ 1.0 đến 0.1 trong suốt quá trình huấn luyện. Bộ đệm trải nghiệm replay duy trì dung lượng 50.000 chuyển đổi 
để đảm bảo trải nghiệm học tập đa dạng.

\textbf{Chiến Lược Phối Hợp:} Hệ thống sử dụng cách tiếp cận đa tác nhân phân cấp với phối hợp tập trung 
để tối ưu hóa hiệu suất toàn mạng. Chia sẻ thông tin giữa các tác nhân nút giao thông xảy ra thông qua 
máy chủ trung tâm tạo điều kiện cho việc ra quyết định được phối hợp. Huấn luyện được đồng bộ hóa trên tất cả 
4 tác nhân nút giao thông để đảm bảo động lực học tập nhất quán, trong khi chuyên môn hóa mô hình nhận thức 
độ phức tạp giao thông cho phép hiệu suất tối ưu trên các điều kiện giao thông đa dạng.

\subsection{Các Chỉ Số Đánh Giá}\label{subsec2b-2}

Hiệu suất hệ thống được đánh giá sử dụng một tập hợp toàn diện các chỉ số nắm bắt cả khía cạnh hiệu quả 
và chất lượng dịch vụ của điều khiển giao thông:

\subsubsection{Chỉ Số Hiệu Quả Giao Thông}

\textbf{Thời Gian Chờ Trung Bình:} Thời gian trung bình xe dành để đứng yên tại các nút giao thông, 
được đo bằng giây mỗi xe. Chỉ số chính này phản ánh trực tiếp hiệu quả điều khiển giao thông và chất lượng 
trải nghiệm người dùng.

\textbf{Độ Dài Hàng Đợi:} Số lượng xe trung bình chờ đợi tại tín hiệu giao thông, được chuẩn hóa theo 
dung lượng làn đường. Độ dài hàng đợi chỉ ra việc sử dụng dung lượng nút giao thông và mức độ tắc nghẽn tiềm năng.

\textbf{Thông Lượng:} Tổng số xe được xử lý mỗi giờ trên toàn mạng, đo lường dung lượng tổng thể hệ thống 
và hiệu quả trong việc xử lý nhu cầu giao thông.

\textbf{Thời Gian Di Chuyển:} Thời gian hành trình từ đầu đến cuối cho xe đi qua mạng, bao gồm cả thành phần 
thời gian di chuyển và chờ đợi. Chỉ số này phản ánh hiệu suất tổng thể mạng từ góc độ người dùng.

\subsubsection{Chỉ Số Chất Lượng Dịch Vụ}

\textbf{Phương Sai Độ Trễ:} Độ lệch chuẩn của thời gian chờ đợi trên các xe khác nhau và khoảng thời gian 
khác nhau, chỉ ra tính nhất quán của hệ thống và sự công bằng trong cung cấp dịch vụ.

\textbf{Tỷ Lệ Dừng:} Phần trăm xe được yêu cầu dừng tại các nút giao thông, phản ánh độ mượt mà của luồng 
giao thông và hiệu quả thời gian tín hiệu.

\textbf{Phần Thưởng Tích Lũy:} Tổng phần thưởng nhận được trong quá trình huấn luyện của mô hình DQN, 
phản ánh khả năng học tập và cải thiện hiệu suất của hệ thống qua thời gian.

\subsubsection{Chỉ Số Phối Hợp}

\textbf{Chỉ Số Đồng Bộ Hóa Pha:} Đo lường phối hợp thời gian giữa các nút giao thông liền kề, được tính toán 
như hệ số tương quan giữa các mô hình thời gian tín hiệu.

\textbf{Cân Bằng Luồng Mạng:} Đánh giá phân phối giao thông trên toàn mạng, đo lường khả năng của hệ thống 
ngăn chặn tắc nghẽn cục bộ thông qua cân bằng tải.

\subsection{Các Phương Pháp So Sánh Cơ Sở}\label{subsec2b-3}

Hệ thống đa nút giao thông được đồng bộ hóa đề xuất được đánh giá so với hai cách tiếp cận đã được thiết lập 
để chứng minh lợi ích phối hợp:

\subsubsection{Cơ Sở Thời Gian Cố Định (Không Tối Ưu Hóa)}

Điều khiển tín hiệu giao thông thời gian cố định truyền thống với các mô hình thời gian cơ bản phục vụ 
như cơ sở hiệu suất. Cách tiếp cận này đại diện cho hoạt động tín hiệu giao thông không đổi, không được 
tối ưu hóa đặc trưng bởi thời gian chờ trung bình không đổi 45.0 giây và độ dài hàng đợi trung bình 9.5 xe. 
Hệ thống thiếu khả năng học tập thích ứng hoặc tối ưu hóa, phục vụ như điểm tham chiếu cho các tính toán 
cải thiện trong phân tích so sánh.

\subsubsection{DQN Nút Giao Thông Đơn Lẻ}

Các bộ điều khiển DQN riêng lẻ hoạt động độc lập tại mỗi nút giao thông mà không có cơ chế phối hợp 
phục vụ như cơ sở thứ hai. Cách tiếp cận này chứng minh tầm quan trọng của thành phần phối hợp phân cấp 
bằng cách tiết lộ khả năng tối ưu hóa cục bộ trong các nút giao thông riêng lẻ trong khi làm nổi bật 
tiềm năng cải thiện hạn chế do thiếu phối hợp cấp mạng. Nó thiết lập giới hạn hiệu suất trên cho các hệ thống 
thông minh không được phối hợp và cung cấp cơ sở so sánh trực tiếp để đo lường lợi ích đồng bộ hóa.

\subsection{Kết Quả Thí Nghiệm}\label{subsec2b-4}

Đánh giá toàn diện chứng minh hiệu quả của cách tiếp cận đa nút giao thông được phối hợp trên các kịch bản 
độ phức tạp giao thông khác nhau.

\subsubsection{Hiệu Suất Tổng Thể Hệ Thống}

Hệ thống 4 nút giao thông được đồng bộ hóa đạt được cải thiện đáng kể so với cả cách tiếp cận cơ sở 
và nút giao thông đơn lẻ:

\begin{table}[h]
\centering
\caption{So Sánh Hiệu Suất Tổng Thể Hệ Thống}
\label{tab:overall_performance}
\begin{tabular}{@{}lccc@{}}
\toprule
\textbf{Hệ Thống} & \textbf{Thời Gian Chờ (s)} & \textbf{Độ Dài Hàng Đợi (xe)} & \textbf{Cải Thiện} \\
\midrule
Cơ sở (Thời gian cố định) & 45.0 & 9.5 & - \\
DQN Nút Giao Thông Đơn Lẻ & 38.5 & 8.5 & 14.3\% \\
4-Nút Giao Thông Đồng Bộ & 32.9 & 7.3 & 27.0\% \\
\bottomrule
\end{tabular}
\end{table}

Kết quả chứng minh rằng hệ thống được đồng bộ hóa đạt được cải thiện 27.0\% trong việc giảm thời gian chờ 
so với cơ sở không được tối ưu hóa, đại diện cho lợi ích bổ sung 12.5\% so với tối ưu hóa nút giao thông đơn lẻ.

\subsubsection{Phân Tích Tác Động Độ Phức Tạp Giao Thông}

Một phát hiện quan trọng của nghiên cứu này là mối tương quan mạnh mẽ giữa độ phức tạp giao thông 
và hiệu quả tối ưu hóa. Hiệu suất thay đổi đáng kể trên bốn kịch bản giao thông:

\begin{table}[h]
\centering
\caption{Tác Động Độ Phức Tạp Giao Thông đến Hiệu Suất Hệ Thống}
\label{tab:complexity_analysis}
\begin{tabular}{@{}lcccc@{}}
\toprule
\textbf{Kịch Bản} & \textbf{Lưu Lượng Giao Thông} & \textbf{Cơ Sở} & \textbf{Hiệu Suất Cuối} & \textbf{Cải Thiện} \\
\midrule
Giao Thông Thấp & 300 xe/h & 28.0s & 18.3s & 34.5\% \\
Giao Thông Trung Bình & 600 xe/h & 42.0s & 31.5s & 25.1\% \\
Giao Thông Cao & 900 xe/h & 58.0s & 45.8s & 21.1\% \\
Giờ Cao Điểm & 1200 xe/h & 68.0s & 59.6s & 12.4\% \\
\bottomrule
\end{tabular}
\end{table}

Phân tích này tiết lộ mối tương quan nghịch mạnh mẽ (r = -0.95) giữa độ phức tạp giao thông và thành công 
tối ưu hóa, cung cấp thông tin chi tiết quan trọng cho các chiến lược triển khai thực tế.

\subsubsection{Phân Tích Hội Tụ Huấn Luyện}

Quá trình huấn luyện chứng minh sự hội tụ ổn định trên tất cả các kịch bản, với tốc độ hội tụ thay đổi 
theo độ phức tạp giao thông. Điều kiện giao thông thấp (300 xe/h) thể hiện sự hội tụ nhanh chóng vào tập 80 
với sự hình thành plateau sớm. Các kịch bản giao thông trung bình (600 xe/h) cho thấy cải thiện ổn định 
với sự hội tụ đạt được vào tập 110. Điều kiện giao thông cao (900 xe/h) chứng minh các mô hình học tập dần dần 
với sự hội tụ vào tập 130. Các kịch bản giờ cao điểm (1200 xe/h) thể hiện sự hội tụ chậm hơn vào tập 140 
kèm theo biến động cao hơn trong động lực học tập.

Tất cả các lần chạy huấn luyện đều hội tụ thành công trong giai đoạn huấn luyện 150 tập, chứng minh tính mạnh mẽ 
của cách tiếp cận đề xuất trên các mức độ phức tạp khác nhau.

\subsection{Cân Nhắc Triển Khai Hệ Thống}\label{subsec2b-5}

Kết quả thí nghiệm cung cấp thông tin chi tiết thực tế cho việc triển khai thực tế hệ thống điều khiển 
giao thông thông minh.

\subsubsection{Chiến Lược Chuyên Môn Hóa Mô Hình}

Dựa trên phân tích độ phức tạp giao thông, hệ thống tạo ra các mô hình chuyên môn hóa được tối ưu hóa 
cho các điều kiện giao thông khác nhau. Mô hình giao thông thấp (300 xe/h) được tối ưu hóa cho phản ứng 
nhanh chóng và hiệu quả trong các kịch bản mật độ thấp, trong khi mô hình giao thông trung bình (600 xe/h) 
cung cấp cách tiếp cận cân bằng cho điều kiện giao thông đô thị điển hình. Mô hình giao thông cao (900 xe/h) 
tích hợp khả năng phối hợp được tăng cường cho các giai đoạn tắc nghẽn, và mô hình giờ cao điểm (1200 xe/h) 
cung cấp xử lý chuyên môn hóa cho các kịch bản độ phức tạp tối đa.

\subsubsection{Xác Thực Hiệu Suất}

Cách tiếp cận phối hợp đồng bộ hóa chứng minh lợi ích nhất quán trên tất cả các kịch bản được kiểm tra. 
Đồng bộ hóa cung cấp cải thiện bổ sung 12.7\% so với tối ưu hóa nút giao thông đơn lẻ, trong khi tính ổn định 
huấn luyện được chứng minh bởi sự hội tụ thành công trong 100\% các lần chạy huấn luyện trong 150 tập. 
Hệ thống thể hiện khả năng mở rộng hiệu quả thông qua phối hợp trên mạng 4 nút giao thông với quản lý tập trung, 
và duy trì cải thiện hiệu suất mạnh mẽ trên tất cả 4 mức độ phức tạp giao thông riêng biệt.

\subsubsection{Hiệu Quả Tính Toán}

Hệ thống chứng minh yêu cầu tính toán thực tế phù hợp cho triển khai thời gian thực. Kích thước mô hình 
nhỏ gọn chỉ có 6.626 tham số mỗi tác nhân nút giao thông, đảm bảo sử dụng bộ nhớ hiệu quả. Thời gian huấn luyện 
vẫn thực tế với sự hội tụ đạt được trong 150 tập trên tất cả các kịch bản. Yêu cầu bộ nhớ được tối ưu hóa 
thông qua dung lượng bộ đệm replay 50.000 chuyển đổi hiệu quả, trong khi thiết kế tổng thể hệ thống đảm bảo 
phù hợp cho triển khai trên phần cứng tính toán tiêu chuẩn cho hoạt động thời gian thực.

Những kết quả này chứng minh rằng hệ thống đề xuất cung cấp một giải pháp thực tế và hiệu quả cho điều khiển 
giao thông thông minh với lợi ích có thể đo lường được trên các điều kiện giao thông đa dạng.

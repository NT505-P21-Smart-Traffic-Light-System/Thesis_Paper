\section{Introduction}\label{sec1}

Urban traffic congestion has emerged as one of the most pressing challenges facing modern metropolitan areas worldwide. The rapid urbanization and increasing vehicle ownership have resulted in significant transportation bottlenecks, leading to extended travel times, increased fuel consumption, and elevated air pollution levels. Ho Chi Minh City exemplifies this global crisis, experiencing a 17\% increase in traffic congestion during the first quarter of 2025, with over 9.4 million registered vehicles competing for limited road infrastructure. The air quality deterioration has become particularly alarming, with PM2.5 concentrations reaching 80μg/m³—four times the World Health Organization's recommended limit of 20μg/m³.

Traditional traffic management systems predominantly rely on fixed-time signal control, where traffic lights operate according to predetermined schedules regardless of actual traffic conditions. While such systems provide predictable operations, they fundamentally lack the adaptivity required to respond to dynamic traffic patterns, emergency situations, or varying demand throughout the day. Modern actuated control systems represent an improvement by detecting vehicle presence through sensors, yet they remain limited to reactive responses based on immediate local conditions without considering network-wide optimization or coordination between adjacent intersections.

The advent of artificial intelligence and Internet of Things (IoT) technologies presents unprecedented opportunities to revolutionize traffic management through intelligent, adaptive, and coordinated control systems. Deep Reinforcement Learning (DRL) has demonstrated remarkable success in complex decision-making scenarios, offering the potential to learn optimal traffic control policies through interaction with dynamic environments. Unlike traditional optimization approaches that require explicit mathematical models of traffic behavior, DRL agents can adapt and improve their decision-making capabilities based on observed outcomes and changing conditions.

This paper presents a comprehensive intelligent traffic light control network system that integrates DRL with IoT technologies to address the limitations of conventional traffic management approaches. The proposed system employs a hierarchical architecture that combines local Deep Q-Network (DQN) agents for individual intersection control with a global Soft Actor-Critic (SAC) agent for network-wide synchronization. Computer vision capabilities utilizing YOLO11 enable real-time vehicle detection and tracking from existing surveillance infrastructure, providing accurate traffic state information without requiring extensive additional sensor deployment.

The primary contributions of this research include: (1) development of a practical DRL-based traffic control system that achieves 14.3\% improvement in waiting time reduction at individual intersections compared to fixed-time control; (2) implementation of a synchronized multi-intersection coordination mechanism demonstrating 27.0\% overall improvement with an additional 12.5\% benefit over single intersection optimization; (3) comprehensive analysis of the relationship between traffic complexity and DRL optimization effectiveness, revealing strong inverse correlation (r = -0.95) with practical implications for deployment strategies; and (4) generation of five specialized TensorFlow models optimized for different traffic scenarios, providing ready-to-deploy solutions for real-world implementation.

The experimental validation demonstrates that the proposed system achieves competitive performance compared to established commercial solutions while offering superior adaptability and easier deployment. The research provides evidence-based guidance for practical implementation, identifying optimal deployment scenarios and resource requirements for different traffic complexity levels. This work contributes to the growing body of research in intelligent transportation systems, offering a viable path toward more efficient and sustainable urban mobility solutions.
